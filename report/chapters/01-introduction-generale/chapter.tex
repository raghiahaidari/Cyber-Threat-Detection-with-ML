\chapter*{Introduction}
\addcontentsline{toc}{chapter}{Introduction}

\thispagestyle{empty}

In our course of Machine Learning, we embarked on a project to apply and expand our knowledge. This endeavor served as a practical application of the concepts and techniques we've learned, while also serving as a platform to explore new horizons in the field. 
Our focus was on developing a model for detection of Distributed Denial of Service (DDoS) attacks, a critical aspect of cybersecurity. 
 
This report details our journey, from the initial exploration of DDoS attacks and PCAP files to the training of our model and the development of an application for real-world deployment.

The journey began with a comprehensive exploration of DDoS attacks. We also studied the structure of PCAP files, understanding how they capture and store network traffic data. This foundational knowledge was crucial in preparing our dataset and guiding our feature engineering process.

The dataset we utilized was carefully selected to represent a diverse range of network traffic scenarios, including both normal and attack traffic. Through data exploration, we gained valuable insights into the characteristics of DDoS attacks, which informed our approach to model development.

Our focus then shifted to training the Random Forest model, a process that involved fine-tuning hyperparameters and optimizing feature selection. We also experimented with different approaches to feature engineering, seeking to improve the model's ability to detect subtle patterns indicative of DDoS attacks.

Finally, we developed an application for our trained model, allowing users to upload their PCAP files and receive an analysis of the file.


\pagestyle{default}